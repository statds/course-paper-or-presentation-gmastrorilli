\documentclass[12pt]{article}

%% preamble
\usepackage{amsmath}
\usepackage[margin = 1in]{geometry}
\usepackage{graphicx}
\usepackage{booktabs}
\usepackage{natbib}

% highlighting hyper links and refrences 
\usepackage[colorlinks=true, citecolor=blue]{hyperref}

% Need to research how to properly cite websites


\title{Influences of GDP in the U.S }
\author{Ginamarie Mastrorilli\\
  Department of Statistics\\
  University of Connecticut
}

\begin{document}
\maketitle


\paragraph{Abstract}


\paragraph{Keywords}



\paragraph{Introduction}
%  the introduction sections need to explain the importance of the topic of the paper, provide the background of the research work, and highlight the contributions of the work. At the end of the introduction, a roadmap, or an outline of the paper is useful in helping the readers navigating through the following sections.
%Why does it matter?
%What have been done?
%What is new?

% start with overview 

What is Gross Domestic Product? Gross Domestic Product is one of the main factors that goes into determining a countries economic growth. 
GDP is the total monetary value of all goods and services that are produced in a country and is a "comprehensive measure of U.S economic activity" \citet[]{bea}.
Gross Domestic Product was originally invented in the 1600's but evolved into governmental use in the 1900's. 
GDP became a national tool to measure a countries economic activity in the 1940's after the Bretton Woods confrence in New Hampshire, US.
At this time, Gross National Product was still a main tool to measure production, but in 1991, the United States swiched to using GDP as its main estimate. 
GDP is calculated by 

$C + I + G + NX = GDP$

summing private consumption, gross private investments, government investments, government spending, and the difference between exports and imports (Net Exports). 
This equation is one that is learned in into to macroeconomics courses. 
Calculating GDP is something that economists have 


There has been extensive research into GDP and what factors impact it. 
Which factors have been researched are subjective based on those who are conducting the study.



% existing works 
\citet{divya2014study} found that exchange rates and market indexes are important factors that influence an economies GDP. They also found that inflation is highly correlated, but not a significant influencer.  

Szustak found that the relationship between power production and GDP is random. This 

% roadmap and contributions

For this paper, we will quantify the relation between influencers of GDP. 






\paragraph{Data}



\paragraph{Methods}




\paragraph{Application}




\paragraph{Discussion}


\paragraph{Appendix}


\bibliography{refrences}
\bibliographystyle{chicago}

\end{document}