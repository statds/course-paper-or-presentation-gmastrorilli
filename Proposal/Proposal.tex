\documentclass[12pt]{article}

%% preamble: Keep it clean; only include those you need
\usepackage{amsmath}
\usepackage[margin = 1in]{geometry}
\usepackage{graphicx}
\usepackage{booktabs}
\usepackage{natbib}

% for space filling
\usepackage{lipsum}
% highlighting hyper links
\usepackage[colorlinks=true, citecolor=blue]{hyperref}


%% meta data

\title{Proposal: What Impacts GDP in the U.S.? }
\author{Ginamarie Mastrorilli\\
  Department of Statistics\\
  University of Connecticut
}

\begin{document}
\maketitle


\paragraph{Introduction}
%Introduction: Introducing the topic and why you have chosen this topic (3–5 lines). 
%Mention briefly the current related research and cite relevant works.
Gross Domestic Product is one of the main factors that goes into determining a countries economic growth. Gross Domestic Product or GDP is a "comprehensive measure of U.S economic activity." (add bea citation) 
(cite K. Hema Divya) founnd that exchange rates and market indexes are important factors that influence an economies GDP. They also found that inflation is highly correlated, but not a significant influencer.  

\paragraph{Specific Aims}
%Specific aims: Formulate a research question or hypothesis in the chosen topic.
% Describe briefly why you select such a question or hypothesis and its importance in the field (cite sources).
With so many economic factors and data readily avaiable to us, which ones shoudl economists focus on?
When investigating U.S. Gross Domestic Product, the goal is to determine the relationship between variables and if those variables can help predict GDP.
If a variable tends to predict the value of GDP, then economists should focus more on these independent variables when exploring Gross Domestic Product.
I decided to choose this topic because as an economics minor, I have seen similar analysis, but have never had the opportunity to do this statistical analysis myself. 
This is important becuause even though there is an abdunace of resouces avaiable about GDP, this information can change based on country, variables used, and which years your data contains. 


\paragraph{Data}
My dependent variable will be Gross Domestic Product (GDP).
% add more vars or change last variable + add (ex:GDP) to rest of variables 
My independent variables will be Population (Annual Percent), nominal broad u.s. dollar index, 10-Year Treasury Constant Maturity Rate, Disposable Personal Income(Annual Percent), Unemployment Rate(Annual Percent), and Labor Productivity: Business Sector. 

The data is collected from the Federal Reserve Economic Data (FRED) website. All of the datasets have been collected seperately and compiled into one Excel file. 

\paragraph{Research Design and Methods}
% say statisical methods
For this paper, I plan to create a multiple linear regression model. 
For this model, my dependent variable will be Gross Domestic Product, and my independent variables will be ...(add). 
My goal is to find which independent variables are associated and can help predict the value of GDP. 



\paragraph{Discussion}
\lipsum[5] \citep{wild2004global}

\paragraph{Conclusion}
\lipsum[1]


\bibliography{../manuscript/refs}
\bibliographystyle{chicago}

\end{document}
