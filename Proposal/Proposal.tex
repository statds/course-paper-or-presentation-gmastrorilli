\documentclass[12pt]{article}

%% preamble: Keep it clean; only include those you need
\usepackage{amsmath}
\usepackage[margin = 1in]{geometry}
\usepackage{graphicx}
\usepackage{booktabs}
\usepackage{natbib}

% for space filling
\usepackage{lipsum}
% highlighting hyper links
\usepackage[colorlinks=true, citecolor=blue]{hyperref}


%% meta data

\title{Proposal: What Impacts GDP in the U.S.? }
\author{Ginamarie Mastrorilli\\
  Department of Statistics\\
  University of Connecticut
}

\begin{document}
\maketitle


\paragraph{Introduction}
As an economics minor, the economics in the United States is something that has become more interesting to me over the oast few years. I have chosen to investigate this topic becuase it is something that is interesting to me, but is also something I can be successful at investigating. Even though I have taken a lot of economics courses at college, I have never investigated this topic myself or by using statistical methods.

\paragraph{Specific Aims}
\lipsum[2]

\paragraph{Data}
My dependent variable will be Gross Domestic Product (GDP).
% add more vars or change last variable + add (ex:GDP) to rest of variables 
My independent variables will be Population (Annual Percent), Interest Rate(Annual Percent), 10-Year Treasury Constant Maturity Rate, Disposable Personal Income(Annual Percent), Unemployment Rate(Annual Percent), and All Transactions House Price Index. 

The data is collected from the Federal Reserve Economic Data (FRED) website. All of the datasets have been collected seperately and compiled into one document. 

\paragraph{Research Design and Methods}
\lipsum[4]

\paragraph{Discussion}
\lipsum[5] \citep{wild2004global}

\paragraph{Conclusion}
\lipsum[1]


\bibliography{../manuscript/refs}
\bibliographystyle{chicago}

\end{document}
